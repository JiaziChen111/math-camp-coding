\documentclass{beamer}

\mode<presentation>
{
  \usetheme{default}      % or try Darmstadt, Madrid, Warsaw, ...
  \usecolortheme{default} % or try albatross, beaver, crane, ...
  \usefonttheme{default}  % or try serif, structurebold, ...
  \setbeamertemplate{navigation symbols}{}
  \setbeamertemplate{footline}[frame number]
  \setbeamertemplate{caption}[numbered]
} 

\usepackage[english]{babel}
\usepackage[utf8x]{inputenc}

\usepackage{graphicx}
\usepackage{subcaption}
\usepackage{hyperref}

\title{Math Camp 2020: Programming (part 1)}
\author{Frank Pinter}
\date{20 August 2020}

\AtBeginSection[]
{
  \begin{frame}
    \frametitle{Table of Contents}
    \tableofcontents[currentsection]
  \end{frame}
}

\begin{document}

\begin{frame}
  \titlepage
\end{frame}

\begin{frame}{Outline}
  \tableofcontents
\end{frame}

\section*{Introduction}

\begin{frame}{Structure of today's session}
\begin{enumerate}
    \item General programming advice ($\approx 30$ min)\itemsep=2em
    \item Julia walkthrough ($\approx 60$ min)
    \item Open time for questions ($\approx 30$ min)
    \begin{itemize}
        \item Feel free to ask questions throughout!
    \end{itemize}
\end{enumerate}
All materials from today's presentation are on GitHub at \\
\url{https://github.com/fpinter/math-camp-coding}
\end{frame}

\section{Basic principles}

\begin{frame}{Basic principles}
    \begin{enumerate}
        \item Learn by doing (practice!)
        \item Always keep your future self in mind
        \item Your time is valuable
        \item Talk to people about programming
    \end{enumerate}
\end{frame}

\begin{frame}{Learn by doing}
    \begin{itemize}
        \item Especially early in grad school, treat learning about programming as an investment
        \item Practice new skills as often as you can
    \end{itemize}
\end{frame}

\begin{frame}{Always keep your future self in mind}
    \begin{itemize}
        \item When you return to a project later on, you should be able to:\itemsep=1.5em
        \begin{itemize}
            \item Figure out what's going on
            \item Not screw things up
        \end{itemize}
        \item Write clear readme files (don't rely on your memory)
        \item Clearly written code $\gg$ over-commenting
        \begin{itemize}
            \item Use good variable names
            \item Use good function names
            \item Use functions to simplify things
        \end{itemize}
        \item Write comments with a specific audience in mind
        \begin{itemize}
            \item Typically your future self and your collaborators
        \end{itemize}
    \end{itemize}
\end{frame}

\begin{frame}{Your time is valuable}
    \begin{itemize}
        \item Spend time improving your ability to work, organization, and accuracy\itemsep=2em
        \item Do not spend too much time making your code run faster
        \item<2> A classic quote in computing: \begin{quote}
    ``Programmers waste enormous amounts of time thinking about, or worrying about, the speed of noncritical parts of their programs, and these attempts at efficiency actually have a  strong negative impact when debugging and maintenance are considered. We \textit{should} forget about small efficiencies, say about 97\% of the time: premature optimization is the root of all evil.''
    \end{quote}
    --\href{https://doi.org/10.1145/356635.356640}{Donald Knuth}
    \end{itemize}
\end{frame}

\begin{frame}{Talk to people about programming}
    \begin{itemize}
        \item There might be ways to solve your problem you hadn't thought of\itemsep=2em
        \item Talk to your cohort, talk to people you know
        \item Stay up to date on tools
    \end{itemize}
\end{frame}

\section{More specific advice}

% Miklos Koren tweet

\begin{frame}{When writing code}
    \begin{itemize}
        \item Don't repeat yourself\itemsep=2em
        \begin{itemize}
            \item Don't copy and paste; write functions
        \end{itemize}
        \item Write (and save) formal tests
        \begin{itemize}
            \item Write tests for functions \textit{when you write the functions}
            \item Write checks your data should pass
        \end{itemize}
        \item Know and use the idioms of your language
        \begin{itemize}
            \item Know \textit{why} your code works the way it does
            \item Know the common gotchas
            \item Nothing should be magic!
        \end{itemize}
        \item Understand all unexpected results
        \begin{itemize}
            \item Learn how to read the error messages
        \end{itemize}
    \end{itemize}
\end{frame}

\begin{frame}{When organizing your project}
    \begin{itemize}
        \item Aim for full reproducibility, including a detailed readme file instructing a replicator \textit{exactly what to do}\itemsep=2em
        \begin{itemize}
            \item Keep this file continuously updated as you work
            \item Fewer steps for the replicator = better
        \end{itemize}
        \item Don't write critical parts of your code under time pressure
        \begin{itemize}
            \item If you do, go back and clean it up later
        \end{itemize}
        \item Use version control to track changes over time
        \item Split your code into steps, with a clear order
        \begin{itemize}
            \item You should be able to clear all your outputs/intermediate files and run the master script
        \end{itemize}
    \end{itemize}
\end{frame}

\begin{frame}{Note on choosing programming languages}
    \begin{itemize}
        \item Tired: wars between programming languages on Twitter\itemsep=2em
        \item Wired: using the right language for the task at hand
        \item There's no rule saying you have to use the same language for everything (even within a project)
        \item Questions to ask yourself:
        \begin{itemize}
            \item What do your coauthors use?
            \item In what language do you work most efficiently?
            \item What functionality do you need?
            \begin{itemize}
                \item e.g., data cleaning, web scraping, heavy computation
            \end{itemize}
        \end{itemize}
    \end{itemize}
\end{frame}

\section{Specific advice: first year}

\begin{frame}{First year vs. research}
    \begin{tabular}{p{2.5cm}|p{3.5cm}|p{3.5cm}}
    & G1 coursework & Research/real life \\
    \hline
    Day to day work &
Numerical computation with clean or simulated data &
Mostly wrangling real-world data (unless you’re a theorist) \\
\hline
Maintainability &
Submit and you’re done &
Return to your code many times, sometimes years later \\
\hline
Accuracy &
Nice to have &
Be obsessive about making sure your results are correct \\
    \end{tabular}
\end{frame}

\begin{frame}{First year vs. resarch}
    \begin{tabular}{p{2cm}|p{4cm}|p{4cm}}
    & G1 coursework & Research/real life \\
    \hline
Testing &
Smell test + write formal tests for basic debugging &
Smell test + write lots of formal tests \\
\hline
Collaboration &
Discuss with group, but write code independently &
Depends on your style and your coauthors’ styles \\
\hline
Version control &
Nice to have, but optional &
Very important
    \end{tabular}
\end{frame}

\section{Further resources}
\begin{frame}{Further resources}
\begin{itemize}
    \item \href{https://scholar.harvard.edu/ristovska/resources}{LJ Ristovska's presentation}\itemsep=2em
    \item \href{https://www.sas.upenn.edu/~jesusfv/teaching.html}{Jesús Fernández-Villaverde's lecture notes}
    \item \href{https://quantecon.org/}{QuantEcon}
    \item \href{https://www.iq.harvard.edu/data-science-services/workshop-materials}{Harvard IQSS training materials}
\end{itemize}
\end{frame}

\begin{frame}{How to get help}

\begin{enumerate}
    \item Check the built-in help in the language
    \item Google
    \begin{itemize}
        \item Often the result will be a Stack Overflow answer -- these are often helpful but not always
        \item Watch out for out-of-date info (especially for Julia)
    \end{itemize}
    \item Ask someone
    \begin{itemize}
        \item Plug for the econ department Slack
    \end{itemize}
    \item Ask a question on Stack Overflow
    \begin{itemize}
        \item Stack Overflow has guidance on how to ask a good question; read that first
    \end{itemize}
\end{enumerate}
\end{frame}

\section{Julia}

\begin{frame}{Why Julia today?}
\begin{itemize}
    \item The focus of math camp: skills you'll use in first year\itemsep=2em
    \begin{itemize}
      \item Numerical computation, matrix algebra, optimization
    \end{itemize}
    \item Julia excels at these and its matrix syntax is clean
    \item<2-> Historically the dominant language for first year PhD was Matlab
    \begin{itemize}
        \item Julia syntax is closely based on Matlab\itemsep=1em
        \item Unlike Matlab, Julia is free and open-source, with a growing community, and many of the advantages of modern languages
        \item You can switch back to Matlab anytime if you want
    \end{itemize}
\end{itemize}
\end{frame}

\begin{frame}{Alternatives to Julia}
\begin{itemize}
    \item R\itemsep=2em
    \begin{itemize}
        \item De facto standard in statistics
        \item Great for work with real data
        \item Matrix syntax is less intuitive
    \end{itemize}
    \item<2-> Python
    \begin{itemize}
        \item De facto standard in physical sciences, engineering, and the tech industry
        \item Great all-purpose language (``Swiss army knife'')
        \item Matrix syntax has improved but is still less intuitive than Julia's
    \end{itemize}
\end{itemize}
\end{frame}

\begin{frame}{Downsides of Julia}
\begin{itemize}
    \item Generally harder to use than R or Python for manipulation of real data\itemsep=2em
    \item The language is unstable (most help files online before 2018 are useless)
    \item The community is less active than R and Python
\end{itemize}
\end{frame}

\end{document}
